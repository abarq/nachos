\section[Desarrollo]{Desarrollo}
  \subsection{Estrategia de implementación}   
    La estrategia a utilizar para la solución del problema de la implementación de los llamados del sistema fue mediante la modificación de los archivos del systema operativo y la adición de las respectivas funciones en el archivo de exception, esto para poder atrapar los llamados al sistema cuando se ejecute una exception del tipo system call. \\
    Gran parte de la estrategia a seguir fue dad mediante los laboratorios de Nachos 6 , 7 y 8 donde se realiza la separación de los casos de excepciones mediante un switch, la implementación de los llamados al sistema de open, read y write, fork. \\
    Finalmente la creación del archivo de nachosFileTabla para poder tener un control de los archivos que se encontraban abiertos por los procesos activos. \\
    Los demás llamados al sistema fueron implementados mediante la utilización de llamados al sitema de unix y métodos propios del sistema de NachOS como lo son el yield , exit y los llamados al sistema para la utilización de semáforos.
    
  \subsection{Archivos modificados}
    Para la implemenentación de los llamados del sistema fueron necesarios realizar algunas modificaciones a los archivos principales del sistema operativo nachOS, los archivos de exception.cc, addressSpace tuvieron grandes cambios por lo que no se incluyen en el informe.\\
    Sin embargo para el manejo de los hilos para el llamado de Fork fue necesario la utilización de un bitmap y un arreglo de hilos actuales, además para el uso de las páginas de memoria del simulador MIPS se utilizó también un bitmap para las páginas utilizadas. esto se puede ver en el siguiente extracto del archivo threads.h y threads,cc
    \begin{lstlisting}
class Thread{
...
    void Print() { printf("%s, ", name); }
	
    int id;
    BitMap * waitingProcess;
    Semaphore * sem;
    int archivosUsados[MaxArchivos];
    int numArchivosUsados;

  private:
  ...
    \end{lstlisting}
    Luego se realiza la inicialización de estas variables en el constructor de la clase Thread.
    \begin{lstlisting}

Thread::Thread(const char* threadName)
{
    name = threadName;
    stackTop = NULL;
    stack = NULL;
    status = JUST_CREATED;
    openFilesTable = new NachosOpenFilesTable(); 
    waitingProcess = new BitMap(20);
    sem = new Semaphore("semaforo",0);
    numArchivosUsados = 0;
    for(int i = 0; i < 100; i++){
    	archivosUsados[i] = -1;
    }
#ifdef USER_PROGRAM
    space = NULL;
#endif
}
    \end{lstlisting}
     Además de la clase Thread, también se realizaron pequeñas modificaciones en la clase system para poder implementar los llamados al sistema
     \begin{lstlisting}
Class System{
...
  //variables nuevas
  extern BitMap * pageTableMap;
  extern BitMap * bitmapSemaforos;
  extern Semaphore* semaforosActuales[20];
  extern BitMap * hilosMap;
  extern Thread *hilosActuales[MaxHilos]; 
  extern Semaphore* semExec;                          
  extern char nombreEx[100];                          
  extern int *states;                                
...
}     
     \end{lstlisting}
     Estas variables son nuevamente declaradas en el archivo system.cc y luego inicializadas mediante el método initialize, cabe destacar que estas variables son declaradas como globales para todo el sistema operativo por lo que pueden ser accesadas desde cualquier parte del código.

    
